\section{Interface graphique}
L'utilisateur doit pouvoir interagir avec l'application, il faut donc lui offrir un interface graphique qui lui permette d'effectuer les différentes actions possibles. Cela dit, si la cible principale est un smartphone, cela implique que l'espace réservé pour l'interface est restreint. L'idée est donc de disposer de quelque chose de léger, avec les contrôles sur les bords de l'écran (Figure \ref{fig:ui-normal}).

\begin{figure}[H]
	\centering
	\includegraphics[width=0.9\linewidth]{ui-Normal}
	\caption{Interface graphique d'Arc3D.}
	\label{fig:ui-normal}
\end{figure}

L'interface se compose de cinq zones bien distinctes (références couleurs sur la figure \ref{fig:ui-zones}):
\begin{itemize}
	\item En \emph{rouge}, le choix du trajet ou de la destination, ainsi que la possibilité d'activer ou non le mode \textit{personnes à mobilité réduite};
	\item En \emph{vert}, un seul bouton qui permet de passer en mode plein écran et d'en revenir;
	\item En \emph{jaune}, un curseur permettant de modifier la vitesse de déplacement;
	\item En \emph{bleu}, deux boutons pour choisir le mode \textit{Live/Simulation} ainsi que deux boutons de mise en marche/pause/arrêt;
	\item Le fond qui est notre canevas WebGL.
\end{itemize}

\begin{figure}[H]
	\centering
	\includegraphics[width=0.9\linewidth]{ui-Zones}
	\caption{Les différentes zones de l'application.}
	\label{fig:ui-zones}
\end{figure}

La dernière fonctionnalité de l'interface n'est pas graphique. Par dessus le canevas WebGL se trouvent deux zones cliquables: une à gauche et une à droite (Figure \ref{fig:ui-leftright}). Elles permettent d'effectuer la <<calibration>> du gyroscope à la main, c'est-à-dire d'ajuster la direction de la caméra dans la scène, en décalant respectivement à gauche ou à droite.

\begin{figure}[H]
	\centering
	\includegraphics[width=0.9\linewidth]{ui-LeftRight}
	\caption{Les deux zones cliquables qui permettent de décaler le gyroscope.}
	\label{fig:ui-leftright}
\end{figure}