\section{Rendu graphique}
Afin de rendre notre image à l'utilisateur, il existe différentes manières de calculer la couleur d'un objet en tout point. La version la plus primitive est de simplement afficher la texture que l'on a appliqué dessus, le problème est que c'est loin d'être agréable à regarder. C'est pour cela qu'il a été découvert plusieurs techniques afin de simuler un comportement des couleurs et des lumières afin d'obtenir un rendu plus réaliste.

\subsection{Type de matériaux}
Quand on parle de matériaux dans \textit{threejs} on va en fait définir de quelle manière se comporte le Shader \cite{wiki-shader} et sur quel modèle d'illumination on se base. Afin de gagner du temps et des performances j'ai décidé d'utiliser les matériaux, donc Shaders, de \textit{threejs} car ces derniers sont exhaustifs et déjà optimisés. Deux choix s'offrent alors à nous: soit le matériaux de Lambert (Shading de Gouraud \cite{wiki-gouraud} et un modèle d'illumination de Lambert \cite{wiki-lambert}), soit un matériaux de Phong (Shading et modèle d'illumination de Phong \cite{wiki-phong-shading} \cite{wiki-phong-reflectance}).

Pour des raisons de performance sur mobile, c'est le modèle de Lambert qui a été choisi. Les rendus sont moins photo-réalistes, mais il est bien plus important d'avoir plus de 30 images par seconde sur mobile.


\subsection{Éclairage}
L'éclairage de la scène est limité, les modèles d'illuminations demandent de faire une moyenne de toutes les sources de lumières en chaque point. Ce qui ne pose aucun problème sur des ordinateurs de bureau, mais les smartphone ne sont pas tous capable de calculer cela tout en assurant les 30 images par seconde. On doit se contenter d'une lumière ambiante, une directionnelle et quelques sources ponctuelles. 