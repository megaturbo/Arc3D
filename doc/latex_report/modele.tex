\section{Modèle 3D}
Si l'on veut permettre à l'utilisateur de se repérer dans le bâtiment, il faut en premier lieu obtenir un modèle de ce dernier. Un projet d'étudiant a dans le passé permis de générer une première version de ce dernier à l'aide du logiciel \emph{Blender} \cite{blender-website}. Cependant  après l'avoir inspecté, il s'est avéré qu'il comportait beaucoup de polygones inutiles, voir Figure \ref{fig:model-toomanypolygons}.

\begin{figure}[h!]
	\centering
	\includegraphics[width=0.9\linewidth]{model-TooManyPolygons}
	\caption{Un exemple des défauts du modèle}
	\label{fig:model-toomanypolygons}
\end{figure}

A ce problème existe deux solutions: soit effectuer un \textit{geometry clean up} qui consiste à supprimer le surplus de polygone, soit modéliser à nouveau le bâtiment. Sachant que je ne possède aucune connaissance en modélisation il a été jugé plus sage de modéliser une nouvelle fois l'école. Ceci dans un but d'apprentissage plus logique si l'on commence par les bases.

L'environnement de modélisation choisi est \emph{Autodesk 3ds Max} \cite{autodesk-3dsmax}. Ce logiciel professionnel offre beaucoup plus de fonctionnalités que Blender et une licence étudiant gratuite, ce qui fait de cela une bonne solution pour la modélisation.


\subsection{Modélisation géométrique}
Pour modéliser correctement l'école, il est important de se munir des plans du bâtiment, car effectuer toutes les mesures à la main est source d'erreur et de perte de temps. Les seuls plans qu'il a été possible de récupérer sont les plans de sols, les hauteurs ont donc été calculées par ratio. Ici le but n'est pas la précision exacte du modèle mais de se réaliser où l'on se trouve, ce n'est donc pas un problème d'avoir un quelconque manque de précision sur les hauteurs des étages.

Premièrement, la forme des murs a été reportée depuis le plan (Figure \ref{fig:model-formemur}) sur \textit{Autodesk 3ds Max} (Figure \ref{fig:model-formemurautodesk}). Ensuite la forme des murs a pu être extrudé afin d'obtenir des objets 3D, voir Figure \ref{fig:model-murextrude}. A partir de ce modèle il a commencé à être possible de pouvoir travailler la véritable forme du bâtiment. C'est à dire modéliser les pas de portes et fenêtres, les sols et plafonds ainsi que les cages d'escaliers (Figure \ref{fig:model-onefloor} et \ref{fig:model-onestairs}). Pour obtenir au final un modèle complet du Campus Arc 2 de la Haute-Ecole Arc de Neuchâtel (Figure \ref{fig:model-all}).

\begin{figure}
\centering
\includegraphics[width=0.9\linewidth]{model-FormeMur}
\caption{Plans du bâtiment utilisés pour la modélisation.}
\label{fig:model-formemur}
\end{figure}

\begin{figure}
\centering
\includegraphics[width=0.9\linewidth]{model-FormeMurAutodesk}
\caption{Le résultat après le report du plan sur \textit{Autodesk 3ds Max}.}
\label{fig:model-formemurautodesk}
\end{figure}

\begin{figure}
	\centering
	\includegraphics[width=0.9\linewidth]{model-MurExtrude}
	\caption{Les murs du bâtiment après extrusion.}
	\label{fig:model-murextrude}
\end{figure}

\begin{figure}
	\centering
	\includegraphics[width=0.9\linewidth]{model-OneFloor}
	\caption{Un étage du bâtiment.}
	\label{fig:model-onefloor}
\end{figure}

\begin{figure}
	\centering
	\includegraphics[width=0.9\linewidth]{model-OneStairs}
	\caption{Un escalier du bâtiment.}
	\label{fig:model-onestairs}
\end{figure}

\begin{figure}
	\centering
	\includegraphics[width=0.9\linewidth]{model-All}
	\caption{Le modèle du bâtiment complet.}
	\label{fig:model-all}
\end{figure}


\subsection{Texturisation du modèle}
Maintenant que nous avons un modèle du bâtiment, nous voulons que les utilisateurs le reconnaissent. L'étape de texturisation sert à cela, on va appliquer des textures qui correspondent aux véritables matériaux afin de faire ressembler le modèle au maximum à son origine. 


\subsection{Exportation et importation}
Afin de pouvoir utiliser notre modèle avec threejs il faut employer un format de fichier pour pouvoir le transférer. Plusieurs formats ont étés analysés et expérimentés, celui qui est le mieux supporté par threejs est le format \emph{JSON}\cite{wiki-json}. Il est pourtant simple d'exporter depuis \textit{Autodesk 3ds Max} en format \emph{OBJ}, cependant threejs supporte mal le grand nombre de polygone avec ce format (voir Figure \ref{fig:import-objfail}).

\begin{figure}
	\centering
	\includegraphics[width=0.9\linewidth]{import-ObjFail}
	\caption{threejs supporte mal les modèles composés de beaucoup de polygones en format OBJ.}
	\label{fig:import-objfail}
\end{figure}

\textit{Autodesk 3ds Max} ne permet pas de base d'exporter un modèle en format JSON. Il est néanmoins doté d'un système de plugins qu'il est possible de créer et d'ajouter au logiciel. Les développeur de \textit{threejs} ont déjà pensés à ça et mettent à disposition un plugin permettant d'exporter notre modèle au format JSON. Cependant la librairie étant encore en alpha, tout n'est pas toujours à jour, il a donc fallu modifier le code (écrit en Maxscript \cite{autodesk-maxscript}) afin qu'il corresponde aux derniers standards de la librairie et que l'on puisse l'utiliser sans problème. Du côté WebGL, la librairie offre des utilitaires permettant de charger ces fichiers facilement.