\selectlanguage{francais}
\begin{abstract}
	
	A retravailler.
	
	Le Campus Arc 2 de la Haute-École Arc, (HE-Arc), Neuchâtel, HES-SO, est un bâtiment de grande envergure dont certaines zones ont dues être sécurisées. Ces open-spaces dont l'accès est limité engendrent des problèmes aux visiteurs tout aussi bien qu'aux étudiants. Il est donc parfois difficile de savoir quel chemin emprunter. Pour résoudre ce problème un outil permettant de faciliter les déplacements permettant la visualisation et le tracé du chemin pour se rendre à n'importe quelle salle de ce campus.
	 aux visiteurs, plusieurs passages du bâtiment sont limités à certaines personnes et il est parfois difficile de savoir quel chemin emprunter. Ce rapport décrit le développement d'un outil permettant de faciliter les déplacements de la gare au campus et entre deux endroits du bâtiment.
	
	% plus tard
	Il est a noter que le point de départ peut tout aussi bien être un endroit dans le bâtiment ou même le hall de la gare de Neuchâtel.
	
	La solution apportée se présente sous la forme d'une page Internet accessible depuis un smartphone ou un ordinateur. La technologie utilisée est WebGL ce qui évite tout les problèmes de librairies externes ou de plugins. Le but de l'utilisation de cet outil est avant tout d'apporter une aide aux utilisateurs.
	
\end{abstract}

\selectlanguage{english} 
\begin{abstract}
	Translation will be made when the french part is confirmed.
\end{abstract}