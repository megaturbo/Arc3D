\selectlanguage{francais}
\begin{abstract}
	
	Le Campus Arc 2 de la Haute-École Arc, (HE-Arc), Neuchâtel, HES-SO, est un bâtiment de grande envergure dont certaines zones ont dû être sécurisées. Des open-spaces, dont l'accès est réglementé et limité, engendrent des problèmes tant pour les visiteurs que pour les étudiants. Il est régulièrement difficile de savoir, selon le type d'utilisateur, quel chemin emprunter. Un outil facilitant les déplacements, en permettant la visualisation et le tracé du chemin pour se rendre en tout lieu et en toute salle du campus, pourrait résoudre ce problème. Ce rapport décrit le développement dudit outil, lequel facilite les déplacements à l'intérieur du bâtiment mais englobe également les déplacements depuis la gare au campus. %En résumé, le point de départ peut tout aussi être un endroit dans le bâtiment qu'un autre se trouvant dans la gare de Neuchâtel.
	
	La solution apportée se présente sous la forme d'une page Internet accessible depuis un smartphone ou un ordinateur. La technologie utilisée est celle du WebGL, ce qui évite tous les problèmes de librairies externes ou de plugins.
	
\end{abstract}

\selectlanguage{english} 
\begin{abstract}
	Karim halp me pls.
\end{abstract}