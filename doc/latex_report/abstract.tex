\selectlanguage{francais}
\begin{abstract}
	
	Le Campus Arc 2 de la Haute-École Arc, (HE-Arc), Neuchâtel, HES-SO, est un bâtiment de grande envergure dont certaines zones ont dû être sécurisées. Des open-spaces, dont l'accès est réglementé et limité, engendrent des problèmes tant pour les visiteurs que pour les étudiants. Il est régulièrement difficile de savoir, selon le type d'utilisateur, quel chemin emprunter. Un outil facilitant les déplacements, en permettant la visualisation et le tracé du chemin pour se rendre en tout lieu et en toute salle du campus, pourrait résoudre ce problème. Ce rapport décrit le développement dudit outil, lequel facilite les déplacements à l'intérieur du bâtiment mais englobe également les itinéraires depuis la gare au campus. En résumé, le point de départ peut tout aussi être un endroit dans le bâtiment qu'un autre se trouvant dans la gare de Neuchâtel. La solution apportée se présente sous la forme d'une application web 3D accessible depuis un smartphone ou un ordinateur. La technologie utilisée est celle du WebGL, ce qui évite tous les problèmes de portabilité et d'installation.
	
\end{abstract}

\selectlanguage{english} 
\begin{abstract}
	The Campus Arc 2 of the Haute Ecole Arc, (HE-Arc), Neuchâtel, HES-SO is a relatively large building that may be quite confusing for the newcomer. It contains multiple open spaces with regulated and restricted access that may cause problems for both visitors and students; it is difficult to know, depending on the person, which path they should take to get from one place to another within the building. A potential solution to this problem: a tool that allows the visualization of the building and the path that would get the user from a key point of the building to another. This report describes the development of said tool which also provides pathing from the train station to the campus. The solution presents itself under the form of a 3D web app accessible through a smartphone or computer. The technology used to perform this project is WebGL which avoids problems with installation process, and portability.
\end{abstract}
\selectlanguage{francais}