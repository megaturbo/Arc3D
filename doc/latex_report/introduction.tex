\chapter{Introduction}

\section{Problématique générale}
Au XXIe siècle, tout va toujours plus vite, l'être humain n'entend pas perdre de temps inutilement et, notamment, pas pour chercher son chemin. Pour se déplacer sur un site d'une certaine complexité et d'une certaine envergure, il convient d'offrir une aide au déplacement, sous la forme d'une solution rapide et facile d'accès. L'utilisation de la 3D en navigateur n'est que très peu répandue pour l'instant, mais elle pourrait tout à fait répondre à ce besoin. Proposer un tel outil, innovant et performant, aurait également l'avantage de répondre à une certaine ambition du domaine Ingénierie de l'HE-Arc de Neuchâtel, en matière de visualisation en temps réel.   


\section{Contextualisation}
Dans le cadre du Campus Arc 2, le deuxième étage est considéré comme un open space. La particularité est que différents secteurs sont fermés aux visiteurs et élèves. Outre sa dimension, c'est une des causes principales de problèmes de déplacements à l'intérieur du bâtiment. Il est ainsi fondamentalement intéressant d'offrir un outil pour faciliter les trajets à l'intérieur du bâtiment et depuis la gare.