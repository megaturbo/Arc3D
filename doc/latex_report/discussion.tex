\section{Conclusion}
Arc3D permet de trouver son chemin entre deux salles du Campus Arc 2 de la Haute-École Arc. Il s'agit d'un outil facilement accessible depuis un téléphone portable, une tablette ou encore un ordinateur. La localisation de l'utilisateur, malheureusement pas suffisamment précise, n'a cependant pas pu être mise en place pour des raisons technologiques. Le comblement de cette lacune apporterait la touche finale au projet actuel, qui aurait véritablement pu épater l'utilisateur et faciliter de façon intéressante, agréable et même ludique, l'utilisation du système.

Ce projet est une preuve de concept, certains objectifs ayant pu être atteints, d'autres partiellement atteints et certains non atteints. Cela dit, les différentes approches abordées dans le cadre de ce projet ont apporté la preuve qu'un outil de ce genre est utile pour les visiteurs d'un bâtiment dont la complexité et l'envergure sont importantes.

\section{Perspectives}
Si l'on veut pouvoir offrir une localisation en intérieur, il faut utiliser une autre plateforme. Pour l'instant, une application web ne propose pas les outils nécessaires à la conception de cette fonctionnalité dans sa totalité. Le problème est que si l'on utilise une autre plateforme, par exemple une application native, cela nécessite une installation. Il est néanmoins évident que l'on ne peut pas attendre de n'importe quel utilisateur qu'il soit enclin et capable d'installer une nouvelle application, postulant qu'il ne pourrait être de passage dans le bâtiment en question qu'une seule fois ou que cette installation pourrait lui prendre du temps. 

Comme autre prise et afin de gagner en réalisme, il pourrait être intéressant de travailler sur les lumières de l'environnement. Pour cela, il faudrait énormément optimiser la manière de calcul. Il pourrait, par exemple, être plus performant de pré-calculer les lumières à l'avance.

Comme autre solution, l'utilisateur pourrait disposer d'une meilleure contextualisation dans le bâtiment si on avait placé du mobilier dans les salles, des arbres devant l'école ou encore les textures des bâtiments environnants. Si toutes ces adjonctions pouvaient être réalisées, il serait nécessaire de se focaliser à nouveau sur le rendu en temps réel. Mais être sûr qu'un téléphone mobile peut rendre plus de 30 images par seconde doit être une priorité.