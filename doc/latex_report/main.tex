\documentclass[a4paper,12pt,notitlepage]{report}

% Packages
\usepackage[utf8]{inputenc}
\usepackage[T1]{fontenc} 
\usepackage[francais,english]{babel}
\usepackage{graphicx}
\usepackage[export]{adjustbox}
\usepackage[babel]{csquotes}
\usepackage{enumitem}
\usepackage{hyperref}
\usepackage{float}
\usepackage{geometry}
\geometry{left=2in, right=1in}

\usepackage[%
	 backend=bibtex      % biber or bibtex
	 %,style=authoryear    % Alphabeticalsch
	 ,style=numeric-comp  % numerical-compressed
	 ,sorting=none        % no sorting
	 ,sortcites=true      % some other example options ...
	 ,block=none
	 ,indexing=false
	 ,citereset=none
	 ,isbn=true
	 ,url=true
	 ,doi=true            % prints doi
	 ,natbib=true         % if you need natbib functions
	 ]{biblatex}
	 
\makeatletter
\def\blx@maxline{77}
\makeatother
	
\addbibresource{biblio.bib}


%\setcounter{tocdepth}{1}
\setcounter{secnumdepth}{1}

\graphicspath{{images/}}

\begin{document}

\begin{titlepage}
	\centering
	\includegraphics[width=0.6\textwidth]{hearc_logo}\par
	\vspace{3cm}
	
	{\Huge\bfseries ARC3D\par}
	\vspace{0.5cm}
	{Travail de Bachelor - 16dlm-tb-219\par}
	
	\vfill
	réalisé par\par
	Thomas \textsc{Roulin}
	
	\vspace{0.5cm}
	encadrement pédagogique par\par
	Stéphane \textsc{Gobron}
	\vfill
	
	% Bottom of the page
	{\today\par}
\end{titlepage}
% Abstract in both languages
\vspace*{\fill}
\begin{abstract}
	Résumé du rapport ici.
\end{abstract}
\vspace*{\fill}
\selectlanguage{francais}

% ToC
\tableofcontents

% Content
\chapter{Introduction}

\section{Problématique générale}
Au XXIe siècle, tout va toujours plus vite, l'être humain n'entend pas perdre de temps inutilement et, notamment, pas pour chercher son chemin. Pour se déplacer sur un site d'une certaine complexité et d'une certaine envergure, il convient d'offrir une aide au déplacement, sous la forme d'une solution rapide et facile d'accès. L'utilisation de la 3D en navigateur n'est que très peu répandue pour l'instant, mais elle pourrait tout à fait répondre à ce besoin. Proposer un tel outil, innovant et performant, aurait également l'avantage de répondre à une certaine ambition du domaine Ingéniérie de l'HE-Arc de Neuchâtel, en matière de visualisation en temps réel.   

% Proposé par STG. A voir comment insérer
% De plus notre école a aussi une certaine prétention, notamment dans le domaine de la visualisation en temps réel.




\section{Contextualisation}
Dans le cadre du Campus Arc 2, le deuxième étage est considéré comme un open space. La particularité est que différents secteurs sont fermés aux visiteurs et élèves. Outre sa dimension, c'est une des causes principales de problèmes de déplacements à l'intérieur du bâtiment. Il est ainsi fondamentalement intéressant d'offrir un outil pour faciliter les trajets à l'intérieur du bâtiment et depuis la gare.

\chapter{Analyse}
\section{État de l'art}
Avant de démarrer le développement, il est important de se renseigner sur les travaux qui ont déjà été effectués concernant différents objectifs du projet. Le but est d'examiner si des recherches ou outils déjà créés pourraient nous aider à développer notre projet.


\subsection{3D temps réel en navigateur}
Dans le cadre de ce projet, il n'est pas possible de préparer à l'avance tous les différents chemins imaginables. C'est pourquoi nous devons nous orienter vers la 3D en temps réel. Le but de cette technique est de pouvoir rendre des images rapidement pour que l'œil humain ait l'impression d'une certaine fluidité. Ceci implique le rendu d'au minimum 30 images par seconde, voire 60 au mieux. Il n'est probablement pas utile de faire plus.

La technologie indiquée WebGL est maintenant supportée par la majorité des navigateurs, mobiles compris \cite{caniuse-webgl}. Elle offre un pont entre notre page internet et la carte graphique de l'utilisateur, c'est grâce à cela qu'il est possible d'offrir ce nombre d'images par seconde. 

Cependant, le langage à utiliser est très primitif, c'est pourquoi il est intéressant d'avoir une architecture permettant de gérer une scène. Plusieurs bibliothèques WebGL offrent déjà des infrastructures ainsi que différents outils qui simplifient des tâches qui seraient redondantes et n'apporteraient rien au projet.

Le choix de la bibliothèque s'est porté sur \emph{three.js}\cite{threejs-website} car c'est celle qui est la plus développée et mise à jour, tout en offrant un grand nombre de fonctionnalités. Elle gère notamment le chargement de modèles et permet aussi d'effectuer, par exemple, du \textit{Back-face culling} et du \textit{Frustum culling}\cite{wiki-HiddenSurfaceDetermination}.


\subsection{Navigation dans un bâtiment}
La navigation en intérieur alimente bon nombre de recherches, c'est un sujet sur lequel il n'existe apparemment pas de solution parfaite à l'heure actuelle. Il y a plusieurs manières d'offrir des résultats plus ou moins précis pour répondre à ce problème. La section \ref{sec:localisation} approfondit ce thème et décrit les différentes approches expérimentées.


\subsection{Visualisation architecturale}
Beaucoup de visualisations architecturales existantes se contentent de pré-générer une vidéo dans un bâtiment. De ce fait, les technologies utilisées ne sont pas les mêmes, car cela ne se fait pas en temps réel. D'autres optent pour une série de photographies parmi lesquelles il est possible de naviguer. 

Il existe malgré tout plusieurs produits de visualisation en temps réel. Cependant, la plupart requièrent un logiciel installé sur l'appareil du client et, de plus, ne sont pas utilisables sur mobiles. Dans les rares produits qui utilisent WebGL, il y a par exemple Shapespark \cite{architecture-shapespark}, qui n'en est encore qu'au stade de \emph{pre-release} et qui semble plutôt fournir un service qui offre un produit fini pour navigateur, donc rien d'utilisable pour le développement. PlayCanvas \cite{playcanvas-architecture} permet, de son côté, d'avoir une application en navigateur, mais l'outil est, d'une part, payant et, d'autre part, un moteur 3D qui propose bien plus de fonctionnalités que celles dont nous avons besoin.
\section{Localisation en intérieur}
\label{sec:localisation}

Le plus grand défi de ce projet est la localisation en intérieur de l'utilisateur, par là, le degré de précision de cette localisation. De plus, il est intéressant et agréable pour l'utilisateur de disposer d'un suivi régulier de sa position durant la visualisation 3D. Cette section explique les diverses pistes empruntées ainsi que leurs résultats.

\subsection{GPS}
Qui dit localisation pense GPS (\textit{Global Positioning System}), système qui pourrait, à première vue, présenter une bonne solution. Cependant, cette technologie manque de précision, principalement en intérieur. La précision des GPS mobiles se situerait à peu près entre 4 et 50 mètres, dépendant des conditions et du smartphone utilisé. En général, l'erreur est due aux éléments entre l'utilisateur et le satellite, c'est pourquoi la localisation en intérieur ne peut pas se fier au GPS.

\subsection{Triangulation Wifi}
Une méthode utilisée pour la localisation en intérieur est la triangulation / trilatération, en utilisant les \textit{Access Points} WiFi. Le concept consiste à retenir à l'avance les coordonnées de tous les AP. Chaque AP possède une adresse MAC qui les différencie les uns des autres. Ensuite, il faut scanner les AP depuis l'objet que l'ont veut localiser et, en fonction des forces des AP, il est possible de retrouver la position.

L'avantage de cette solution est l'absence d'infrastructures à mettre en place, étant donné que le Campus est déjà équipé de bon nombre d'AP. Cependant, le but de ce projet est d'offrir une application dans un navigateur Internet. Ainsi, cette contrainte  empêche l'accès à toutes les informations de l'appareil concerné par son utilisation. Malgré les technologies qu'offre ce mode, il est encore impossible de récupérer les informations nécessaires à la triangulation / trilatération depuis une page web.

\subsection{Triangulation Bluetooth}
Une autre manière d'utiliser la triangulation est d'installer des beacons bluetooth. Il s'agit de petits émetteurs placés dans les espaces dans lesquels une localisation est nécessaire. Le fonctionnement de la triangulation est identique au WiFi.


Malheureusement, la mise en place de cette infrastructure est un désavantage à cette solution et, de plus, ces accès bluetooth ne sont pas suffisamment supportés depuis une page web. Le tableau ci-dessous \ref{caniuse-bluetooth}, tiré du site www.caniuse.com, montre que seul Chrome peut le supporter, tout en précisant qu'il faut activer le drapeau nécessaire. Cela démontre que l'API Web Bluetooth n'est pas utilisable pour une application destinée au grand public.

\begin{figure}[h!]
	\center
	\includegraphics[width=10cm,frame]{caniuse_bluetooth}
	\caption{Support de l'API Web Bluetooth par les différents navigateurs.}
	\label{caniuse-bluetooth}
\end{figure}

\subsection{Accéléromètre et Gyroscope}
L'approche par accéléromètre et gyroscope est différente, car elle implique d'utiliser les capteurs internes du smartphone pour calculer une position. HTML5 fournit une API pour détecter l'orientation et les déplacements du dispositif. Autrement dit, l'accès à l'accéléromètre et au gyroscope est possible depuis un navigateur Internet.

A l'aide de l'accélération, il est théoriquement possible de calculer une position. Cependant, les valeurs de ces capteurs comportent un bruit et, pour calculer une position, il faut double-intégrer l'accélération. Cela implique qu'il faut double-intégrer le bruit et cela va créer un \textit{drift}. En quelques secondes, l'erreur peut s'élever à une vingtaine de centimètres. En outre, le gyroscope peut avoir une valeur erronée, due à la suppression de la gravité. En d'autres termes, une erreur d'un degré sur la valeur du gyroscope représente plusieurs mètres d'erreur en quelques secondes \cite{google-SensorFusion}. 

Le but de cette approche était fondamentalement d'estimer la position de l'utilisateur, sans pour autant connaître la valeur exacte. Les démarches et réflexions prouvent qu'une approximation n'est pas envisageable avec ces capteurs.

\chapter{Développement}
\section{Modèle 3D}
Explication des différents problèmes rencontrés avec le modèle, son texturing, son exportation et importation.

\subsection{Modélisation géométrique}
Citer Kevin ?
Seulement les plans de sols et pas de coupe. Approximation par ratio des hauteurs.

\subsection{Texturisation du modèle}
Pourquoi? Contextualisation. se rendre compte qu'on est dans l'école
Application des textures.
Erreur export de fichier, modification du script d'exportation.

\subsection{Exportation et importation}
Type d'objet utilisé.`Contraintes/avantages.
\section{Rendu graphique}
Afin de rendre notre image à l'utilisateur, il existe différentes manières de calculer la couleur d'un objet en tout point. La version la plus primitive est de simplement afficher la texture que l'on a appliqué dessus, le problème est que c'est loin d'être agréable à regarder. C'est pour cela qu'il a été découvert plusieurs techniques afin de simuler un comportement des couleurs et des lumières afin d'obtenir un rendu plus réaliste.

\subsection{Type de matériaux}
Quand on parle de matériaux dans \textit{threejs} on va en fait définir de quelle manière se comporte le Shader \cite{wiki-shader} et sur quel modèle d'illumination on se base. Afin de gagner du temps et des performances j'ai décidé d'utiliser les matériaux, donc Shaders, de \textit{threejs} car ces derniers sont exhaustifs et déjà optimisés. Deux choix s'offrent alors à nous: soit le matériaux de Lambert (Shading de Gouraud \cite{wiki-gouraud} et un modèle d'illumination de Lambert \cite{wiki-lambert}), soit un matériaux de Phong (Shading et modèle d'illumination de Phong \cite{wiki-phong-shading} \cite{wiki-phong-reflectance}).

Pour des raisons de performance sur mobile, c'est le modèle de Lambert qui a été choisi. Les rendus sont moins photo-réalistes, mais il est bien plus important d'avoir plus de 30 images par seconde sur mobile.


\subsection{Éclairage}
L'éclairage de la scène est limité, les modèles d'illuminations demandent de faire une moyenne de toutes les sources de lumières en chaque point. Ce qui ne pose aucun problème sur des ordinateurs de bureau, mais les smartphone ne sont pas tous capable de calculer cela tout en assurant les 30 images par seconde. On doit se contenter d'une lumière ambiante, une directionnelle et quelques sources ponctuelles. 
\section{Recherche de chemin}

\subsection{Nœuds}
Format
Placement dans l'espace

\subsection{Algorithme}
A-Star
\section{Contrôles de la caméra}
\subsection{Ordinateur et mobile}
Les différences à gérer
\subsection{Orientation mobile}
Utilisation du gyroscope
\section{Interface graphique}
L'utilisateur doit pouvoir interagir avec l'application, il faut donc lui offrir un interface graphique qui lui permette d'effectuer les différentes actions possibles. Cela dit, si la cible principale est un smartphone, cela implique que l'espace réservé pour l'interface est restreint. L'idée est donc de disposer de quelque chose de léger, avec les contrôles sur les bords de l'écran (Figure \ref{fig:ui-normal}).

\begin{figure}[H]
	\centering
	\includegraphics[width=0.9\linewidth]{ui-Normal}
	\caption{Interface graphique d'Arc3D.}
	\label{fig:ui-normal}
\end{figure}

L'interface se compose de cinq zones bien distinctes (références couleurs sur la figure \ref{fig:ui-zones}):
\begin{itemize}
	\item En \emph{rouge}, le choix du trajet ou de la destination, ainsi que la possibilité d'activer ou non le mode \textit{personnes à mobilité réduite};
	\item En \emph{vert}, un seul bouton qui permet de passer en mode plein écran et d'en revenir;
	\item En \emph{jaune}, un curseur permettant de modifier la vitesse de déplacement;
	\item En \emph{bleu}, deux boutons pour choisir le mode \textit{Live/Simulation} ainsi que deux boutons de mise en marche/pause/arrêt;
	\item Le fond qui est notre canevas WebGL.
\end{itemize}

\begin{figure}[H]
	\centering
	\includegraphics[width=0.9\linewidth]{ui-Zones}
	\caption{Les différentes zones de l'application.}
	\label{fig:ui-zones}
\end{figure}

La dernière fonctionnalité de l'interface n'est pas graphique. Par dessus le canevas WebGL se trouvent deux zones cliquables: une à gauche et une à droite (Figure \ref{fig:ui-leftright}). Elles permettent d'effectuer la <<calibration>> du gyroscope à la main, c'est-à-dire d'ajuster la direction de la caméra dans la scène, en décalant respectivement à gauche ou à droite.

\begin{figure}[H]
	\centering
	\includegraphics[width=0.9\linewidth]{ui-LeftRight}
	\caption{Les deux zones cliquables qui permettent de décaler le gyroscope.}
	\label{fig:ui-leftright}
\end{figure}
\section{Rendu temps réel}

Notre objectif est d'avoir un rendu en temps réel, c'est à dire une fréquence d'image au minimum de 30 par secondes. Si tout le bâtiment devrait être calculé à chaque image il est clair que nous aurions pas un rendu assez rapide. Comme expliqué dans le chapitre \emph{Analyse} de ce rapport, la librairie WebGL que l'on utilise (three.js) offre du \textit{Backface et Frustum culling}. Grâce à ces avantages il est déjà plus simple d'effectuer un rendu temps réel.

Une solution imaginée pour gagner en performance était d'effectuer une découpe du bâtiment en de nombreux morceaux (Schéma en figure \ref{fig:rendering-slice}). Chacune de ces pièces serait sauvegardée dans deux versions différentes: une en haute qualité et l'autre en moins bonne qualité. A partir de là l'idée était de mettre en place un système \emph{LoD (Level of Detail)} \cite{wiki-lod}. Ce qui signifie qu'on charge le modèle en bonne qualité seulement si la caméra est proche de ce dernier, sinon on charge le modèle en qualité plus basse. Cependant les tests effectuées n'ont pas montrés un réel gain de performance. Il a été conclu que ceci était du au \textit{culling} géré par la three.js.

\begin{figure}
\centering
\includegraphics[width=0.7\linewidth]{rendering-Slice}
\caption{Schéma d'une découpe du bâtiment possible.}
\label{fig:rendering-slice}
\end{figure}

\subsection{Textures indexées}
Un problème de performance s'est montré après avoir exporté la version texturée du modèle. Afin d'appliquer plusieurs textures sur un même objet avec \textit{Autodesk 3ds max} il faut utiliser des \emph{Multi/Sub-Object Material}. C'est en fait une liste de matériaux indexés, on l'applique à notre objet et ensuite on définit quel indice du matériau la face affiche. Cependant après avoir utilisé ce genre de matériau sur le modèle, le nombre d'image par seconde sur mobile est descendu aux alentours de deux ou trois. L'analyse a fait remarquer que le \textit{renderer} est appelé environs 5000 fois et sans les textures il est appelé 30 à 40 fois. Le problème vient en fait que pour chaque fragment il doit changer de texture et c'est cette partie qui lui prend énormément de temps. La solution adopté est de classer notre géométrie par textures, comme cela le \textit{renderer} ne doit changer qu'un nombre limité de fois de textures et le nombre d'image par seconde est de nouveau en dessus de 30.

\subsection{Matériel utilisé}
L'application offre au minimum 30 images par seconde. Mais le nombre d'images par seconde n'est pas très objectif sans les informations du matériel utilisé. Les tests effectués sur un ordinateur portable se sont fait avec un \emph{HP EliteBook 8570w}:

\begin{description}[align=right, labelwidth=3cm]
	\item [OS] Microsoft Windows 10
	\item [Chipset]	Mobile Intel® QM77 Express Chipset
	\item [CPU] 3rd Generation Intel® CoreTM i7 Quad-Core
	\item [GPU] A NVIDIA Quadro K2000M
\end{description}

Pour les tests sur smartphone c'est un \emph{Motorola Nexus 6} qui a été utilisé:

\begin{description}[align=right, labelwidth=3cm]
	\item [OS] 	Android OS v6.0 (Marshmallow)
	\item [Chipset]	Qualcomm Snapdragon 805
	\item [CPU] Quad-core 2.7 GHz Krait 450
	\item [GPU] Adreno 420
\end{description}


\chapter{Résultats}
Ici nous discuterons des différents objectifs et de leur réussite ou non. Ceci dans le but de comprendre pourquoi certains objectifs ont abouti contrairement à d'autres.

\section{Modélisation géométrique}


\section{Temps réel}
Un des points importants d'une visualisation dans le genre d'Arc3D est d'avoir une certaine fluidité. Il a fallu faire attention aux calculs envoyés à la carte graphique afin que l'œil humain ne voit pas de coupure entre deux images rendues. Grâce à différentes astuces présentées dans le chapitre \emph{Développement}, le produit final offre au minimum, sur le matériel utilisé, 30 images par secondes, ce qui remplit l'objectif.

\section{Balade de navigation}
Plusieurs mécanique rentre en compte sur ce point, premièrement Arc3D offre une recherche de chemin optimal entre deux endroits du bâtiment ainsi que depuis la gare.
Durant la balade qui parcourt le chemin trouvé il est important que l'utilisateur ait l'impression de se déplacer dans le bâtiment. La caméra effectue des mouvements qui sont linéaires et pourraient ressembler au déplacement d'un humain. Le caméra regarde quelques mètres devant elle afin que l'utilisateur puisse anticiper le chemin.


\section{Réalisation des objectifs}

\chapter{Discussion}
\section{Conclusion}
\section{Perspectives}
Objectifs pas réussis : Comment les réussir.

Améliorations possibles.

Pas de perspectives faciles.


%\chapter{Bibliographie}
\nocite{*} %Even non-cited BibTeX-Entries will be shown.
\printbibliography

\end{document}