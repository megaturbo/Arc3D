\section{État de l'art}
Avant de démarrer le développement, il est important de se renseigner sur les travaux qui ont déjà été effectués concernant différents objectifs du projet. Le but est d'examiner si des recherches ou outils déjà créés pourraient nous aider à développer notre projet.


\subsection{3D temps réel en navigateur}
Dans le cadre de ce projet, il n'est pas possible de préparer à l'avance tous les différents chemins imaginables. C'est pourquoi nous devons nous orienter vers la 3D en temps réel. Le but de cette technique est de pouvoir rendre des images rapidement pour que l'œil humain ait l'impression d'une certaine fluidité. Ceci implique le rendu d'au minimum 30 images par seconde, voire 60 au mieux. Il n'est probablement pas utile de faire plus.

La technologie indiquée WebGL est maintenant supportée par la majorité des navigateurs, mobiles compris \cite{caniuse-webgl}. Elle offre un pont entre notre page internet et la carte graphique de l'utilisateur, c'est grâce à cela qu'il est possible d'offrir ce nombre d'images par seconde. 

Cependant, le langage à utiliser est très primitif, c'est pourquoi il est intéressant d'avoir une architecture permettant de gérer une scène. Plusieurs bibliothèques WebGL offrent déjà des infrastructures ainsi que différents outils qui simplifient des tâches qui seraient redondantes et n'apporteraient rien au projet.

Le choix de la bibliothèque s'est porté sur \emph{three.js}\cite{threejs-website} car c'est celle qui est la plus développée et mise à jour, tout en offrant un grand nombre de fonctionnalités. Elle gère notamment le chargement de modèles et permet aussi d'effectuer, par exemple, du \textit{Back-face culling} et du \textit{Frustum culling}\cite{wiki-HiddenSurfaceDetermination}.


\subsection{Navigation dans un bâtiment}
La navigation en intérieur alimente bon nombre de recherches, c'est un sujet sur lequel il n'existe apparemment pas de solution parfaite à l'heure actuelle. Il y a plusieurs manières d'offrir des résultats plus ou moins précis pour répondre à ce problème. La section \ref{sec:localisation} approfondit ce thème et décrit les différentes approches expérimentées.


\subsection{Visualisation architecturale}
Beaucoup de visualisations architecturales existantes se contentent de pré-générer une vidéo dans un bâtiment. De ce fait, les technologies utilisées ne sont pas les mêmes, car cela ne se fait pas en temps réel. D'autres optent pour une série de photographies parmi lesquelles il est possible de naviguer. 

Il existe malgré tout plusieurs produits de visualisation en temps réel. Cependant, la plupart requièrent un logiciel installé sur l'appareil du client et, de plus, ne sont pas utilisables sur mobiles. Dans les rares produits qui utilisent WebGL, il y a par exemple Shapespark \cite{architecture-shapespark}, qui n'en est encore qu'au stade de \emph{pre-release} et qui semble plutôt fournir un service qui offre un produit fini pour navigateur, donc rien d'utilisable pour le développement. PlayCanvas \cite{playcanvas-architecture} permet, de son côté, d'avoir une application en navigateur, mais l'outil est, d'une part, payant et, d'autre part, un moteur 3D qui propose bien plus de fonctionnalités que celles dont nous avons besoin.