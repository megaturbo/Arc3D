\documentclass{article}
\usepackage[utf8]{inputenc}
\usepackage[T1]{fontenc} 
\usepackage[francais,english]{babel}
\setcounter{secnumdepth}{0}

\begin{document}
	\section{Arc3D - Cahier des charges}
	
	\subsection{Contexte}
	Dans un contexte de communication avec l’extérieur, nous souhaitons développer une image attractive et moderne de notre Haute Ecole Arc.
	
	\subsection{Problématique}
	La recherche du parcours qui mène à une des nombreuses salles de notre établissement est un problème récurrent aussi bien pour les visiteurs, industriels, chargés de cours ponctuels, que pour les nouveaux étudiants.
	
	\subsection{Objectifs principaux}
	Les objectifs principaux fixés pour ce travail de Bachelor sont les suivants:
	\begin{itemize}
		\item Éclairage de base;
		\item Vue à la première personne;
		\item Balade de navigation avec orientation;
		\item Deux modes:
		\subitem Simulation du chemin;
		\subitem Live mode. Suivi de l'utilisateur.
		\item 30 FPS;
		\item Localisation en intérieur et GPS en extérieur.
	\end{itemize}
	
	\subsection{Objectifs secondaires}
	Les objectifs considérés comme secondaires pour ce travail de Bachelor sont les suivants:
	\begin{itemize}
		\item Trouver les toilettes les plus proches;
		\item Tester la vitesse de connexion de l'utilisateur;
		\item Mode Ascenseur pour les personnes à mobilité réduite;
		\item Split de la scène avec Level of Detail;
		\item Mobilier;
		\item Texturisation et lumières dans la scène;
		\item iOS  et Android;
		\item 60 FPS.
	\end{itemize}
	
\end{document}